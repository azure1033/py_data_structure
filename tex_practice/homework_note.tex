\documentclass{ctexart}

\usepackage{graphicx}

\title{数学工具作业}
\author{lin}
\date{\today}

\begin{document}
\maketitle
\section{第一题}

显然有:
\begin{equation}
    \vec{\vec{I}}=\delta_{ij}I_{ij}\partial_{j}\varphi=\delta_{ij}(\partial_{j}\varphi)=\nabla\varphi=\delta_{ij}(I_{ij}\partial_{j}\varphi)
    =\nabla(\vec{\vec{I}}\varphi)
\end{equation}
\section{第二题}
\begin{equation}
    \nabla\cdot(\vec{A}\times\vec{B})=\partial_{i}\varepsilon_{ijk}A_{j}B_{k}=\varepsilon_{ijk}(B_{k}\partial_{i}A_{j}+A_{j}\partial_{i}B_{k})=
    \vec{B}\cdot(\nabla\times\vec{A})-\vec{A}\cdot(\nabla\times\vec{B})
\end{equation}
\section{第三题}
\subsection{(a)}
\begin{equation}
    \nabla\times(\vec{\vec{\omega}\times\vec{x}})=\vec{\omega}(\vec{x}\cdot\nabla)+\vec{\omega}(\nabla\cdot\vec{x})-\vec{x}(\vec{\omega}\cdot\vec{x})-\vec{x}(\nabla\cdot\vec{\omega})
    =3\vec{\omega}-\vec{\omega}=2\vec{\omega}
\end{equation}
注意到$\nabla\cdot\vec{x}=3$且$\nabla\cdot\vec{\omega}=0$.
\subsection{(b)}
由(a)显然有:
\begin{equation}
    \int_{S}2\vec{\omega}\cdot d\vec{a}=\int_{S}[\nabla\times(\vec{\omega}\times\vec{x})]\cdot d\vec{a}=\oint_{\partial V}(\vec{\omega}\times\vec{x})\cdot d\vec{l}
    =\oint_{\partial V}(\vec{x}\times d\vec{l})\cdot\vec{\omega}
\end{equation}
等式两边同时消去$\vec{\omega}$即得到所求等式.
\begin{figure}[htbp]
\centering\includegraphics[width=10cm]{116161847_p0_master1200}
\caption{似乎是碧蓝档案的角色}
\end{figure}

\end{document}
